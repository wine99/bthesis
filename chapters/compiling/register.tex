% !TeX root = ../../main.tex

\section{寄存器分配}

这一步将会把前一节生成的包含变量的x86指令变成真正的x86指令,
包含三个步骤:活变量分析、建立变量之间的干涉图、根据干涉图分配寄存器或内存。

\subsection{活变量分析}

在程序中的某一个点,如果一个变量或寄存器的当前包含的内容在后面会被使用,
我们就说这个变量或寄存器在当前这个点是活的。

我们需要计算出每条指令后所有活变量的集合,
对于不包含分支指令的简单情况,我们从后往前计算按下面的方法计算。

设$I_1,\ldots, I_n$ 为指令序列。
记指令 $I_k$ 之后的活变量集为$L_{\mathsf{after}}(k)$ ,
 $I_k$ 之前为活变量集为 $L_{\mathsf{before}}(k)$ 。
一条指令之后的活变量集与下一条指令之前的活变量集相同。
\begin{equation} \label{eq:live-after-before-next}
L_{\mathsf{after}}(k) = L_{\mathsf{before}}(k+1)
\end{equation}
开始时,在最后一条指令之后没有活变量,所以
\begin{equation}\label{eq:live-last-empty}
L_{\mathsf{after}}(n) = \emptyset
\end{equation}
然后我们重复应用以下规则,将指令序列从后向前遍历。
\begin{equation}\label{eq:live-before-after-minus-writes-plus-reads}
L_{\mathsf{before}}(k) = (L_{\mathsf{after}}(k) - W(k)) \cup R(k)
\end{equation}
其中 $W(k)$ 是指令 $I_k$ 中写入数据的变量或寄存器,
$R(k)$ 是指令 $I_k$ 中读取数据的变量或寄存器。

注意callq 指令应该在其写集 $W$ 中包含所有调用者保存寄存器,
因为这些寄存器可能在函数调用期间被被调用者使用。
同样,callq 指令应当根据调用的函数的参数个数,
在其读集 $R$ 中包含适当的参数传递寄存器。

从显示化控制流这一编译过程的输出开始,编译器的中间形式就从原来的树变成了图。
每个标签下的一系列四元式或者一系列汇编指令,
一直到下一个标签或者到结束,被称作一个基本块。
跳转指令可以从一个基本块跳转到另一个基本块。
二者分别作为有向图的顶点和边,构成了控制流图(CFG)\cite{Allen_1970}。

加入了分支和循环之后,我们要考虑两个问题。第一个问题是跳转指令的活变量分析。
假设第k条指令是跳转指令,它指向某一个基本块,
记该基本块的第一条指令前面的活变量集为$L_{\mathsf{before'}}(1)$。
如果是无条件跳转指令,则
\begin{equation}\label{eq:live-jmp}
L_{\mathsf{before}}(k) = L_{\mathsf{before'}}(1)
\end{equation}
如果是条件跳转指令,则
\begin{equation}\label{eq:live-jmpcc}
L_{\mathsf{before}}(k) = L_{\mathsf{before'}}(1) \cup L_{\mathsf{before}}(k+1)
\end{equation}

第二个问题是以什么样的顺序、重复多少次才能计算出各个基本块中各指令的$L_{\mathsf{after}}$。
解决方法如下:对控制流图进行拓扑排序,选择最末尾的一个基本块开始分析,
计算块内指令的live-after以及块的live-before,也就是第1条指令的live-before,
如果块的live-before中的活变量在这趟分析后增加了,
那么对于所有能跳转这个基本块的块都再进行一次分析,
重复进行下去,直到所有块的live-before集合都不再增大。
这种迭代分析控制流图的方法叫作数据流分析\cite{Kildall_1973},
它也适用于许多其他静态分析问题。
下面的代码描述了这一迭代过程。其中\code{trans-G}是转置后的控制流图,
用以计算哪些基本块指向了某一个块,
\code{label->live}是一个字典,用于记录每个基本块的live-before集合。

\begin{lstlisting}
(while (not (queue-empty? worklist))
  (define label (dequeue! worklist))
  (define-values (new-block new-live-before)
    (uncover-live-block (dict-ref blocks label) label->live))
  (hash-set! new-blocks label new-block)
  (unless (equal? new-live-before (hash-ref label->live label))
    (hash-set! label->live label new-live-before)
    (for ([label^ (in-neighbors trans-G label)])
      (enqueue! worklist label^))))
\end{lstlisting}

\subsection{使用图着色算法进行分配}

知道了任意时刻的活变量集合,我们就能知道某两个变量能不能被分配到同一个寄存器:
如果它们在某一时刻同时是活变量,那我们显然不能把它们分配到同一个寄存器。
把这种情况称之为两个变量互相干涉。我们创建一个无向图,称之为干涉图,
顶点是所有变量,然后在所有干涉的变量之间添加边。

接下来就是为图上的变量分配寄存器,
用从 0 到k − 1 的整数对应于可以用来分配变量的 k 个寄存器。
整数 k 及以上对应的是堆栈位置。
这样一来寄存器分配就变成了图着色问题,这些整数就是可以使用的颜色,
有边的顶点之间不能分配相同的颜色。

本文使用的图着色算法是DSATUR算法\cite{Brélaz_1979}。
其中 SATUR 是饱和度的缩写,它指的是一个顶点上的约束。
我们为某一个顶点涂上一个颜色后,
所有与它相邻的顶点就不能再选用这个颜色,也就多了一个约束。
DSATUR算法每次都选择一个约束最少的顶点,给他涂上编号尽可能小的颜色,
然后给相邻的为上色顶点加上约束,再在剩下的未上色顶点中重复这个过程。

图着色完成后,我们就为所有变量都分配了一个寄存器或者一个栈上的空间。
x86代码已经大致生成完毕,接下来只需要再整理一下一些不合法的x86指令,
再添加上函数的准备工作和结束工作指令即可。
