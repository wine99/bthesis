% !TeX root = ../../main.tex

\section{赋值转换}

考虑这样的例子:

\begin{lstlisting}
(let ([x 2])
  (+ x
     (begin (set! x 40) x)))
\end{lstlisting}

其中 begin 语句接受若干个表达式,依次执行,然后将最后一个表达式的结果返回。
由于在我们的语言中,操作数先依次从左往右被求值,然后执行运算,
上面的代码应该得到 42。

因为汇编语言的运算符的操作数不支持复杂表达式,
而仅仅支持立即数或者从一个寄存器或内存地址中获取操作数。
在后面有一趟被称作原子化操作数的编译过程,会把所有复杂的操作数提取出来,
求值后赋值给一个临时变量,用这个临时变量替换原来的复杂操作数。
像下面这样:

\begin{transformation}
\begin{lstlisting}
(+ (+ x 1) 2)
\end{lstlisting}
\compilesto
\begin{lstlisting}
(let ([tmp (+ x 1)])
  (+ tmp 2))
\end{lstlisting}
\end{transformation}

但是对于上面包含赋值的例子,如果我们只是进行简单的提取替换,会得到下面的结果。

\begin{lstlisting}
(let ([x 2])
  (let ([tmp (begin (set! x 40) x)])
    (+ x tmp)))
\end{lstlisting}

由于对 x 重新赋值的语句原来处于操作数的位置,它会被提取到加法运算的前面。
显然,转换后的程序会得到错误的结果:80。

解决的方法是对于所有涉及到被重新赋值的变量,
我们把对它们的读取包裹在一个复杂操作 get 的后面。
当然,在最终翻译成汇编代码时,通过 get 语句访问变量和直接访问变量并没有区别。
上面的程序将被转换成这样:

\begin{lstlisting}
(let ([x 2])
  (+ (get! x)
     (begin (set! x 40)
            (get! x))))
\end{lstlisting}

这样一来,对这些变量的读取也会被提取到运算之前。
只要原子化操作数这一过程按正确的顺序提取复杂操作数,
我们就能得到正确的程序:

\begin{lstlisting}
(let ([x 2])
  (let ([tmp1 (get! x)])
    (let ([tmp2 (begin (set! x 40)
                       (get! x))])
      (+ tmp1 tmp2))))
\end{lstlisting}

赋值还会导致在编译闭包时出现另一个问题,
我们将在后文描述闭包的实现时指出。
