% !TeX root = ../../main.tex

\section{生成汇编}
\label{sec:x86}

这一步我们会将上一节中生成的四元式序列转换为x86汇编指令序列。
把四元式转换成汇编指令的过程非常直白,
因为四元式只有运算、赋值、函数调用、跳转等少数几种可能的类别,
并且运算的操作数和函数调用的参数都是原子表达式。
唯一的难点在于四元式包含变量,变量名当然是无穷多的,
但x86没有变量,他只有寄存器和过程调用栈。
在生成汇编这趟过程中,我们保留这些变量名,
在下一趟编译过程再把这些变量放到合适的寄存器或栈上去。

接下来我们举一些典型的四元式来说明生成汇编的过程。
首先是整数运算操作,以加法为例:
\begin{transformation}
\begin{lstlisting}
    |$\itm{var}$| = (+ |$\Atm_1$| |$\Atm_2$|);
\end{lstlisting}
\compilesto
\begin{lstlisting}
    movq |$\Arg_1$|, |$\itm{var}$|
    addq |$\Arg_2$|, |$\itm{var}$|
\end{lstlisting}
\end{transformation}

翻译运算操作时要注意可以优化的情况。
比如,如果加法的一个参数与赋值的左边的变量相同,那么就不需要额外的 move 指令。
整个赋值语句可以被翻译成一条 addq 指令:
\begin{transformation}
\begin{lstlisting}
    |$\itm{var}$| = (+ |$\Atm_1$| |$\itm{var}$|);
\end{lstlisting}
\compilesto
\begin{lstlisting}
    addq |$\Arg_1$|, |$\itm{var}$|
\end{lstlisting}
\end{transformation}

read 函数在 x86 汇编中没有直接对应的功能,
所以我们在 runtime.c 中用 C 语言实现一个 \code{read\_int} 函数来提供这个功能。
前文提到,runtime.c 中还包括垃圾回收器以及一些全局变量,
这些功能合在一起被称为“运行时系统”,或简称为“运行时”。
\begin{transformation}
\begin{lstlisting}
      |$\itm{var}$| = (read);
\end{lstlisting}
\compilesto
\begin{lstlisting}
    callq read_int
    movq %rax, |$\itm{var}$|
\end{lstlisting}
\end{transformation}

Return语句的翻译方法和上面两个例子是一样的,只要把赋值左边的var改为\%rax即可。
对于布尔值真和假,我们把它们分别编译成立即数1和0。
然后是布尔运算\code{not}:
\begin{transformation}
\begin{lstlisting}
     |$\itm{var}$| = (not |$\Atm$|);
\end{lstlisting}
\compilesto
\begin{lstlisting}
    movq |$\Arg$|, |$\itm{var}$|
    xorq $1, |$\itm{var}$|
\end{lstlisting}
\end{transformation}
和整数运算一样,这里也需要考察atm来判断能否优化进而减少一条指令。
接下来我们看一下x86对布尔运算和条件分支提供的支持。

cmpq 指令比较它的两个参数来确定一个参数是小于、等于还是大于另一个参数。
cmpq 指令在参数的顺序和结果放置的位置上不太一样。
参数顺序是颠倒的:如果我们想知道 x 是不是小于 y,就需要写 \code{cmpq y, x}。
cmpq 的结果被放置在特殊的 EFLAGS 寄存器中。
这个寄存器不能直接访问,但可以通过许多指令进行查询。
指令 \code{set}cc \code{d} 根据条件代码 cc
(e 表示等于,l 表示小于,le 表示不等于,g 表示大于,ge 表示大于等于),
将立即数 1 或 0 放入目标 d 中。
还需要注意的是d必须是单字节寄存器,例如al或 ah,它们是 rax 寄存器的一部分。
movzbq 指令可以将单个字节寄存器中的值移到 64 位寄存器里。
综上所述,\code{eq?}运算可以被这样翻译:
\begin{transformation}
\begin{lstlisting}
    |$\itm{var}$| = (eq? |$\Atm_1$| |$\Atm_2$|);
\end{lstlisting}
\compilesto
\begin{lstlisting}
    cmpq |$\Arg_2$|, |$\Arg_1$|
    sete %al
    movzbq %al, |$\Var$|
\end{lstlisting}
\end{transformation}

再来看 if 表达式的编译。
指令 \code{j}cc label 更新程序计数器指向 label 之后的指令,
这取决于 EFLAGS 寄存器中的结果是否与条件代码 cc 匹配,
否则条件跳转指令还是继续执行下一条指令。
因为条件跳转指令依赖于 EFLAGS 寄存器,
所以通常它前面会立即有一个 cmpq 指令来设置 EFLAGS 寄存器。

\begin{transformation}
\begin{lstlisting}
if (eq? |$\Atm_1$| |$\Atm_2$|) goto |$\ell_1$|;
else goto |$\ell_2$|;
\end{lstlisting}
\compilesto
\begin{lstlisting}
    cmpq |$\Arg_2$|, |$\Arg_1$|
    je |$\ell_1$|
    jmp |$\ell_2$|
\end{lstlisting}
\end{transformation}

接下来考虑和向量有关的指令。我们保留r11寄存器,专门用它来访问向量也就是堆上的数据。
\begin{transformation}
\begin{lstlisting}
|$\itm{var}$| = (vector-ref |$\itm{vec}$| |$n$|);
\end{lstlisting}
\compilesto
\begin{lstlisting}
  movq |$\itm{vec}$|, %r11
  movq |$8(n+1)$|(%r11), |$\itm{var}$|
\end{lstlisting}
\end{transformation}

\begin{transformation}
\begin{lstlisting}
(vector-set! |$\itm{vec}$| |$n$| |$\itm{arg}$|);
\end{lstlisting}
\compilesto
\begin{lstlisting}
  movq |$\itm{vec}$|, %r11
  movq |$\itm{arg}$|, |$8(n+1)$|(%r11)
\end{lstlisting}
\end{transformation}

\begin{transformation}
\begin{lstlisting}
 |$\itm{vec}$| = (allocate |$n$| |$\itm{type}$|);
\end{lstlisting}
\compilesto
\begin{lstlisting}
  movq free_ptr(%rip), %r11
  addq |$8(n+1)$|,
       free_ptr(%rip)
  movq $|$\itm{tag}$|, 0(%r11)
  movq %r11, |$\itm{var}$|
\end{lstlisting}
\end{transformation}

\begin{transformation}
\begin{lstlisting}
    (collect |$\itm{bytes}$|);
\end{lstlisting}
\compilesto
\begin{lstlisting}
  movq %r15, %rdi
  movq $|\itm{bytes}|, %rsi
  callq collect
\end{lstlisting}
\end{transformation}

\code{free\_ptr}中的地址是FromSpace中的下一个空闲地址,把它赋给\code{r11},
然后把它向前移动$8(n+1)$ 个字节以留出足够的空间来放置向量的内容
(每个元素是8个字节(64位),以及一个64位的标签)。
在编译时我们通过type信息计算出标签,放到向量的第0号位置。
标签的组成参见图~\ref{fig:tuple-rep}处的描述。
对垃圾回收函数collect的调用被编译成对运行时里的collect函数的调用,
参数分别是根栈的顶部地址和需要分配的字节数。
我们还保留了r15寄存器,专门用来保存指向根栈顶部的指针,它会在主程序的准备工作中被初始化。

最后是函数。对于每个函数定义,我们都需要把参数从参数寄存器里移到对应的变量名中:
\begin{transformation}
\begin{lstlisting}
        start:
           |$\itm{instr}_1$|
           |$\vdots$|
           |$\itm{instr}_n$|
\end{lstlisting}
\compilesto
\begin{lstlisting}
    start:
       movq %rdi, |$x_1$|
       movq %rsi, |$x_2$|
       |$\vdots$|
       |$\itm{instr}_1$|
       |$\vdots$|
       |$\itm{instr}_n$|
\end{lstlisting}
\end{transformation}

函数赋值给变量成为 load-effective-address 指令:
\begin{transformation}
\begin{lstlisting}
    |$\itm{var}$| = (fun-ref |$f$|);
\end{lstlisting}
\compilesto
\begin{lstlisting}
    leaq (fun-ref |$f$|), |$\itm{var}$|
\end{lstlisting}
\end{transformation}

然后是函数调用,正如第\ref{sec:call-in-x86}节中所述,它们被编译成x86的间接函数调用:
\begin{transformation}
\begin{lstlisting}
|\itm{var}| = (call |\itm{fun}| |$\itm{arg}_1~\itm{arg}_2\ldots$|));
\end{lstlisting}
\compilesto
\begin{lstlisting}
    movq |$\itm{arg}_1$|, %rdi
    movq |$\itm{arg}_2$|, %rsi
    |$\vdots$|
    callq *|\itm{fun}|
    movq %rax, |\itm{var}|
\end{lstlisting}
\end{transformation}

如果函数调用是尾巴调用,如第\ref{sec:tail-call}节所述,
我们要把上面的callq换成jmp,并在其前面加上弹出当前栈帧的一系列指令,
但由于此时我们还没有进行寄存器分配,也没有生成函数的结束工作,
我们不知道这些指令具体都有哪些,因此我们在这里引入一个临时的\code{TailJmp}指令来替换callq,
后续分配完寄存器之后再把它替换成完整的弹出栈帧和跳转指令。

