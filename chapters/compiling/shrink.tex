% !TeX root = ../main.tex

\section{去糖化}

语法糖是由英国计算机科学家彼得·兰丁发明的一个术语,
指计算机语言中添加的某种语法,这种语法对语言的功能没有影响,
但是更方便程序员使用。
语法糖让程序更加简洁,有更高的可读性。

一个实用的通用编程语言会提供非常多的功能,但从语言实现的角度考虑,
在编译之前将语言的某些部分转换成使用其他更核心的部分来实现,
能让语言维持一个比较小的核心,以便更容易维护同时也更具灵活性。
我们把该步骤称为去糖化,对应的编译过程(pass)称为 shrink。

本文实现的语言中,布尔与、或运算并不属于核心语言,它们会被编译成 if 语句。
\begin{align*}
  \CAND{e_1}{e_2} & \quad \Rightarrow \quad \CIF{e_1}{e_2}{\FALSE{}}\\
  \COR{e_1}{e_2} & \quad \Rightarrow \quad \CIF{e_1}{\TRUE{}}{e_2}
\end{align*}
\begin{comment}
\begin{transformation}
\CAND{e_1}{e_2}
\compilesto
\begin{lstlisting}
\CIF{e_1}{e_2}{\FALSE{}}
\end{lstlisting}
\end{transformation}
\end{comment}


这个转换维持了与、或运算的短路特性。该转换的代码实现如下:

\begin{lstlisting}
(define (shrink-exp e)
  (match e
    [(or (Int _) (Var _) (Bool _) (Void))
     e]
    [(Let x rhs body)
     (Let x (shrink-exp rhs) (shrink-exp body))]
    [(Prim 'and (list e1 e2))
     (If (shrink-exp e1)
         (shrink-exp e2)
         (Bool #f))]
    [(Prim 'or (list e1 e2))
     (If (shrink-exp e1)
         (Bool #t)
         (shrink-exp e2))]
    [(Prim op es) (Prim op (map shrink-exp es))]
    ...
\end{lstlisting}

代码中出现的\code{\#t, \#f}与\code{true, false}相同。
这里处理程序,或者说操作语法树的方法是使用自然递归,也叫结构化递归,
即在结构的子结构上进行递归。
例如,在转换与、或运算时,先对两个运算数进行递归转换,
然后把转换的结果放入 if 表达式中。
这样我们就可以处理任意复杂的与、或运算。
后面的绝大部分编译过程也是如此。
