% !TeX root = ../main.tex

\section{处理函数}

\subsection{区分函数和局部变量}

编译器前端生成的语法树,在没有进行语义分析之前,
没有也无法区分函数名和变量名。
但是对于访问函数,无论是直接调用还是将其赋值给其他变量,
我们需要将其翻译成 leaq 指令,这和访问变量是不同的。
因此我们需要区分开来函数名和变量名。

由于我们已经进行了标识符唯一化,这个工作就变得非常简单。
就像处理可变变量那样,我们引入 Funref 结构体,
把所有出现的使用全局定义的函数名的地方,全部换成 Funref 语句即可。
后面翻译为 x86 时再将其翻译为 leaq 指令。
例如:

\begin{transformation}
\begin{lstlisting}
(define (f) : Integer
  ...)
(let ([g f])
  g)
\end{lstlisting}
\compilesto
\begin{lstlisting}
(define (f) : Integer
  ...)
(let ([g (Funref f)])
  g)
\end{lstlisting}
\end{transformation}

\subsection{生成闭包}

如果一个变量出现在 e 中,但在 e 中没有定义,
那么就称该变量在表达式 e 中是“自由变量”。

例如,在下面的表达式中,x 和 y 为自由变量。
\begin{lstlisting}
(lambda: ([z : Integer]) : Integer
  (+ x (+ y z)))
\end{lstlisting}

而闭包则是引用了自由变量的函数,被引用的自由变量和函数一同存在,
即使已经离开了自由变量的环境也不会被释放或者删除,在闭包中可以继续使用这个自由变量。

例如:
\begin{lstlisting}
(define (f) : (-> Integer)
  (let ([x 0])
    (lambda: () : Integer x)))
((f))
\end{lstlisting}

其中 f 是一个无参函数,在其内部定义了一个局部变量 x,初始值为 0。
调用 f 会返回一个匿名的无参函数,再调用这个匿名函数会返回 x 的值,也就是 0。
显然,我们需要某种方法来使得在执行这个匿名函数时,能正确读取到自由变量 x 的值。




\subsection{赋值对闭包的影响}

考虑下面这个例子:

\begin{lstlisting}
(define (f) : (-> Integer)
  (let ([x 0])
    (let ([g (lambda: () : Integer x)])
      (begin
        (set! x 42)
        g))))
((f))
\end{lstlisting}


