% !TeX root = ../../main.tex

\section{显示化控制流}
\label{sec:control}

这一趟编译过程将把我们的语言翻译成类似四元式的序列,
复杂的控制结构将会变成一些包含跳转语句的简单的语句序列。考虑下面的程序:

\begin{multilinecode}
\begin{lstlisting}
(let ([y (let ([x 20])
           (+ x (let ([x 22]) x)))])
  y)
\end{lstlisting}
\end{multilinecode}

前面描述的编译过程会把程序翻译成左边的结果,
显示化控制流则进一步得到右边的过程:
三
\begin{transformation}
\begin{lstlisting}
  (let ([y
      (let ([x1 20])
        (let ([x2 22])
          (+ x1 x2)))])
    y)
\end{lstlisting}
\compilesto
\begin{lstlisting}
    start:
      x1 = 20;
      x2 = 22;
      y = (+ x1 x2);
      return y;
\end{lstlisting}
\end{transformation}

我们先来考虑最简单的情况:假设程序不包含分支指令,
也不包含可变变量带来的结构,
即只关心最后一个表达式的返回值的\code{Begin}语句和循环语句。
那么程序就只剩下变量定义和运算
(当然还有函数,但是这趟编译过程正是针对每个函数单独处理的,
而函数调用则和运算按同样的方式处理)。
可以看到,就像上面的例子一样,最终的结果就是一系列的赋值语句和一个返回语句。
返回语句是由处于尾位置的表达式编译得到的为。尾位置的定义如下:

(1) \code{(FunctionDef params body)}中,\code{body}处于尾位置。

(2) 如果 \code{(let x e body)} 处于尾位置,则\code{body}处于尾位置。


我们使用两个相互递归的函数explicate-tail 和 explicate-assign 来实现,如图\ref{fig:explicate-control-implementation}所示。

\begin{figure}[t]
\begin{center}
\vspace{0.4em}
\begin{tabular}{c}
\begin{lstlisting}
(define (explicate-tail e)
  (match e
    [(or (Int n))
     (Return (Int n))]
    [(Let x rhs body)
     (let ([instructions (explicate-tail body)])
       (explicate-assign rhs x instructions))]
    ...
    ))

(define (explicate-assign e x cont)
  (match e
    [(Int n)
     (append (Assign (Var x) (Int n))
             cont)]
    [(Let y rhs body)
     (let ([new-cont (explicate-assign body x cont)])
       (explicate-assign rhs y new-cont))]
    ...))
\end{lstlisting}
\end{tabular}
\vspace{0.8em}
\end{center}

    \caption{显示化控制流的实现}
    \label{fig:explicate-control-implementation}
\end{figure}
\begin{multilinecode}
\end{multilinecode}

explicate-tail 函数应用于位于尾位置的表达式,
而 explicate-assign 函数应用于出现在 \key{let} 右侧的表达式。
二者都返回一串四元式。

如果\code{explicate-tail}收到的表达式不是\code{let}表达式,
那么就简单的返回一个四元式\code{return}。
如果是let,那么let中的body部分就属于尾位置,
对其递归调用\code{explicate-tail},得到一串四元式instructions,
把变量名,let语句的右侧,以及这串四元式作为参数传递给\code{explicate-tail}。
\code{explicate-tail}对let语句的右侧表达式进行判断,
如果不是let,那么生成一个赋值四元式,把这个右侧表达式赋给变量名,然后添加到那串四元式的前面即可。
这里用参数名cont来接收四元式序列,意思正是这串四元式将会变成赋值语句的continue。
而如果let的右侧又是一个let,
我们则应当先调用\code{explicate-tail}把这个内部let的body赋给外面的变量,
得到一个新的cont,把它和内部let的变量名和右侧表达式再一次传给\code{explicate-tail}。
就像上面的例子,我们先把\code{(+ x1 x2);}赋给y,\code{return y;}作为其cont,
然后再把这两条语句作为为\code{x1, x2}赋值的语句的cont。
可以说,这一趟编译过程生成四元式序列的过程是倒过来的。
这种实现方式被叫做累积传递风格。

接下来我们考虑上分支语句和赋值,扩展尾位置的定义。
\begin{enumerate}
\item 如果 \code{(if pred then else)} 处于尾位置,
  则表达式\code{then}和\code{else}均处于尾位置。
\item 如果 \code{(begin ... final-e)}处于尾位置,则 \code{final-e}处于尾位置。
\end{enumerate}

与此类似,我们再定义谓词位置和副作用位置,
并相应地定义\code{explicate-pred}和\code{explicate-effect}函数。
\code{explicate-effect}接收cont,
\code{explicate-pred}则接收then-cont和else-cont,
它们也都返回一串四元式序列。
加入分支后,\code{explicate-pred}函数中要生成一些跳转语句,
分别指向then-cont和else-cont。
最终,这趟编译过程将能够完成下面这样的转换:

\begin{transformation}
\begin{lstlisting}
(let ([x (read)])
  (let ([y (read)])
    (if (if (< x 1)
            (eq? x 0)
            (eq? x 2))
        (+ y 2)
        (+ y 10))))
\end{lstlisting}
\compilesto
\begin{lstlisting}
start:
   x = (read);
   y = (read);
   if (< x 1) goto block40;
   else goto block41;
block40:
   if (eq? x 0) goto block38;
   else goto block39;
block41:
   if (eq? x 2) goto block38;
   else goto block39;
block38:
   return (+ y 2);
block39:
   return (+ y 10);
\end{lstlisting}
\end{transformation}
