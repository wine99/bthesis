% !TeX root = ../main.tex

\section{整理指令}

我们生成的x86指令中还存在一些不符合规范的地方。
比如,x86中一条指令最多访存一次,
而前面的编译过程可能会生成包含两次内存引用的代码,
这时我们可以利用保留的rax寄存器做一次中转:

\begin{transformation}
\begin{lstlisting}
movq -8(%rbp), -16(%rbp)
\end{lstlisting}
\compilesto
\begin{lstlisting}
  movq -8(%rbp), %rax
  movq %rax, -16(%rbp)
\end{lstlisting}
\end{transformation}

如果间接函数调用指令的操作数没有被分配到寄存器上,
而是使用了栈空间,我们同样要做一次中转:
\begin{transformation}
\begin{lstlisting}
     callq *-8(%rbp)
\end{lstlisting}
\compilesto
\begin{lstlisting}
  movq -8(%rbp), %rax
  callq *%rax
\end{lstlisting}
\end{transformation}

前面的编译过程会引入很多临时变量,
而图着色算法会把一些变量分配到同一个寄存器或栈位置,
这就可能出现一些movq指令,它的两个操作数是一样的,我们可以直接把这些指令删除掉。

最后就是为每个函数生成第\ref{sec:call-in-x86}节中描述的准备工作和结束工作。
在准备工作中,我们要把所有用到的被调用方保留寄存器存入栈中,
因为这里面可能存放着调用这个函数的调用方的一些数据。
结束工作中再把这些内容逆序弹出回这些寄存器里。
准备工作中还要根据程序中向量的数量移动根栈指针r15。
相应地,在结束工作中也需要把它移动回去。

生成了结束工作后,我们还要把\ref{sec:x86}一节中临时创造的\code{TailJmp}指令换成
真正的间接跳转指令\code{jmp *\%rax},
并在它的前面插入结束工作中除了retq之外的指令。
最后,主程序的准备工作者还需要初始化垃圾回收器,
即调用垃圾回收器的\code{initialize}函数和
把根栈顶部\code{rootstack\_begin}送入根栈指针寄存器r15。

至此,所有的编译工作均已完成,编译器已经能够将一段源程序编译成正确、合法的x86汇编程序,
并且能和runtime.o一起被GCC编译套件链接生成一个能在x86-64平台的Linux系统上运行的可执行文件。
