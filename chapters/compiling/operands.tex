% !TeX root = ../../main.tex

\section{原子化操作数}

前面已经提到过,汇编语言指令的操作数不允许放置复杂表达式,
只能是一个立即数或寄存器或内存位置。
我们通过为每一个复杂的操作数引入一个新的 let 绑定,
将复杂操作数绑定到新变量,然后使用新变量来代替复杂操作数,
就像下面这样。

\begin{transformation}
\begin{lstlisting}
(let ([x (+ 42 (- 10))])
  (+ x 10))
\end{lstlisting}
\compilesto
\begin{lstlisting}
(let ([x (let ([tmp1 (- 10)])
            (+ 42 tmp1))])
   (+ x 10))
\end{lstlisting}
\end{transformation}

本文使用两个相互递归的函数:\code{rco-atom} 和 \code{rco-exp} 来实现这个过程。
其思想是将 \code{rco-atom} 应用于需要成为原子的子表达式,
将 \code{rco-exp} 应用于不需要原子化的子表达式。

\code{rco-exp} 函数根据需要在其子表达式上分别调用
\code{rco-atom} 和 \code{rco-exp}。
\code{rco-atom}同样要对其接受的表达式分类讨论,
继续在复合子表达式上相应的调用\code{(+ 42 (- 10))}或\code{rco-exp}。
\code{rco-atom} 函数会返回两个东西:
一个原子化的表达式(一个变量或者一个数字之类的)
和一个将临时变量映射到原来的复杂操作数的列表。
\code{rco-exp} 根据这个列表创造出一系列的 let 绑定,
包裹在原表达式外面,并把原表达式的那些复杂操作数用原子化后的表达式进行替换。

在上面的例子中,对表达式\code{(+ 42 (- 10))}应用\code{rco-exp}函数,
\code{rco-exp}会对子表达式 \code{42} 和 \code{(- 10)} 分别应用\code{rco-atom}。
前者返回42和空列表,后者返回变量tmp1和把tmp1映射到表达式\code{(- 10)}的列表。

\begin{transformation}
\begin{lstlisting}
            (- 10)
\end{lstlisting}
\compilesto
\begin{lstlisting}
       tmp1
       ((tmp1 . (- 10)))
\end{lstlisting}
\end{transformation}

低效的实现可能会导致冗余变量的产生。
我们要确保\code{rco-exp}只对需要的原子化的子表达式调用\code{rco-atom},
\code{rco-atom}则要确保不要为本身就已经是原子的表达式再创建临时变量。

