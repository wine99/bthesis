% !TeX root = ../main.tex

\chapter{绪论}

\section{课题背景}

当今世界计算机技术飞速发展,各种软件让人眼花缭乱,这得意于程序设计语言的发展。
常用的程序设计语言如 C 语言、C++ 语言、Java 语言等,
用这些高级语言编写程序极大提高了软件开发人员的效率。
但要使这些代码能真正被计算机识别并执行,
需要将它们转化成计算机指令,完成这一工作的便是编译器。
编译器将由高级语言编写的程序翻译成二进制代码或其他目标语言,并在特定平台上运行,
因此程序语言的发展,又很大程度上依赖编译技术的发展。

自上个世纪50年代以来,编译器的相关研究一直是计算机科学领域的一个活跃主题。
从目标机器的角度来说,目标机器处理器从单核到多核,其体系结构在不断的发生变化;
从程序语言的角度来说,高级程序语言在不断的丰富及扩展。

1954 年,John Backus 在 IBM 发明了 FORTRAN。
它是第一个广泛使用的具有功能实现的高级通用编程语言,而不仅仅是纸上的设计。
1957年,他领导的研究小组完成了 FORTRAN 编译器,
该编译器被普遍认为是引入了第一个明确完整的编译器。
1958年,John McCarthy 发明了 Lisp\cite{McCarthy_1960},这是世界上第一门动态类型的函数式编程语言。
六十年代的 Simula 是第一门支持面向对象编程的语言,
其后继 Smalltalk ,一门纯面向对象的语言,在七十年代被发明。
1969 年至 1973 年,C 被发明,用作 Unix 操作系统的系统编程语言,流行至今。
Prolog,第一门逻辑编程语言,在 1972 年被发明。
1978年,ML 在 Lisp 之上建立了一个多态类型系统,开创了静态类型的函数式编程语言。
他们都产生了大量的后代,绝大多数现代编程语言都可以追溯回到这些语言。

作为最早的编程语言之一,Lisp 开创了计算机科学的许多理念,
比如树形数据结构,自动内存管理,动态类型,高阶函数,递归,自举,REPL 等等。
但函数式编程语言作为高度抽象的高级编程语言,与现代计算机的指令之间存在巨大的差别。
而传统的编译器课程只重点讲授教编译器的某几个阶段,如解析、语义分析和寄存器分配等。
这种方法的问题是很难理解整个编译器是如何结合在一起的,
为什么每个阶段是那样设计的,以及那些更具表达力的语言成分究竟是如何实现的。
本文实现了一个简单的 Lisp 方言,将其编译到x86汇编语言,以此来学习并探索编译技术。

\section{国内外研究现状}

Scheme 是一门极简主义 Lisp 方言,采用了词法作用域,
实现了尾递归优化,支持 first-class continuations 等。
1978 年,Guy Steele\cite{Steele_1978} 实现了第一个 Scheme 语言编译器,
该编译器也是第一个使用 CPS 作为中间表示的编译器。
1984 年,Kent Dybvig 创造了第一个商用 Scheme 编译器 Chez Scheme\cite{Dybvig_2006},
它的编译速度很快,生成的目标代码也非常高效。
1992 年,Andrew Appel\cite{Appel_1992} 在其著作中描述了 Standard ML of New Jersey 的实现。
1994年,Christian Queinnec\cite{Queinnec_1996} 在他的书中全面地介绍了 Lisp 语言家族的语义和实现,
给出了 11 个解释器和 2 个目标语言分别为字节码和 C 语言的编译器。
2004 年,Kent Dybvig 等人\cite{Sarkar_Waddell_Dybvig_2004}
在印地安纳大学数年的编译器教学中演化提出了 nanopass 的思想。
2012 年,Andrew Keep\cite{Keep_Dybvig_2013} 在其论文中描述了 nanopass 框架
对商用编译器 Chez Scheme 的改造,论证了该框架的效率并不显著低于传统编译器结构。
2021 年,Jeremy Siek\cite{Siek_2022} 详细总结讲解了编译一个简单的函数式语言到机器语言的完整过程。

\section{本文工作}

本文的工作主要设计并实现了一个简单的 Lisp 方言,将其编译为 x86-64 汇编语言,
二者具体来说分别是 Typed Racket 和 x86-64 的严格子集。
语言支持的功能主要包括可变变量与循环,向量(元组)和垃圾回收,一等函数和闭包。
具体的设计将在“实现语言和源语言”一章中给出。
使用到的 x86-64 指令集将在“目标语言“一章中给出。

编译器后端的主要流程包括去糖化,
变量唯一化,
赋值转换,
函数转换为闭包,
内存管理和垃圾回收,
原子化操作数,
显示化控制流,
指令选择,
寄存器分配,
整理指令。
具体算法流程将在“编译器后端设计实现”一章中给出。

本文还设计实现了一个简单的图形界面,允许我们在输入框中输入一段代码,并排展示出每一趟的编译结果,
输出使用图着色算法时生成的变量相关图以及最终的可执行文件。这一部分将在“展示界面”一章中详细描述。

\section{本章小结}

作为最早的编程语言之一,Lisp 对后续的语言设计和发展产生了极大的影响,整个 Lisp 语言家族也在不断完善和发展。
在“课题背景”一节中,本章首先简要介绍了早期几个重要的编程语言的发展历程,Lisp 语言的地位和特性,
说明了实现一个 Lisp 语言的现实意义与背景。
“国内外研究现状”一节简要介绍了 Lisp 语言及其方言 Scheme 的发展历程,列出了关于其编译器构造的几篇重要文献。
然后指出了本文的编译器实现思想,即 Nanopass 相关的文献。
最后,“本文工作”一节中简要叙述了编译器从源语言到目标语言的各个环节,并给出了对应的章节安排。
