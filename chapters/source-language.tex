% !TeX root = ../main.tex

\chapter{源语言}

在语言特性的选择上,我们主要考虑让它们能尽可能地体现不同的编译原理的知识面。
对于基本数据类型,本文实现了 64 位整数及其加、减、乘、整除、模运算,布尔类型及其与、或运算;
本文没有实现浮点数与浮点运算,因为它们只是单纯地覆盖了更多的 x86-64 指令,
并不会增加语言的表现力;
本文没有采用纯函数式的方式,而是支持了可变变量和循环,
因为这会迫使我们在进行数据流分析时考虑循环数据流;
对于复合数据结构,本文仅实现了元组,这足以让我们考虑堆上的内存数据和垃圾回收;
最后是函数,本文实现了一等函数,利用元组实现了闭包,并且在翻译过程中正确地处理了尾递归。

在类型系统方面,本文没有实现类型推导,但只有函数的形式参数以及返回值需要显式声明类型,
其余变量(即局部变量)定义时并不需要声明类型。例如,如下的代码是一个合法的程序。
\begin{lstlisting}
(define (even? [x : Integer]) : Boolean
  (let ([r (remainder x 2)])
    (if (eq? r 0)
        true
        false)))
\end{lstlisting}

下面给出语言完整的具体语法:

\begin{figure}[h]
  \fbox{
    \begin{minipage}{0.96\textwidth}
      \[
      \begin{array}{rcl}
        \Exp & ::= & \Int \MID \LP\key{read}\RP \MID \LP\key{-}\;\Exp\RP \MID \LP\key{+} \; \Exp\;\Exp\RP \\
        \Lang & ::= & \Exp
      \end{array}
      \]
    \end{minipage}
  }
  \caption{语言的完整具体语法}
  \label{fig:full-con-syntax}
\end{figure}

