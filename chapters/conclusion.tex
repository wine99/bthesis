% !TeX root = ../main.tex

\chapter{总结和展望}

编译技术作为计算机科学领域中一个必不可少的组成部分,
涉及的专业知识繁多,深入研究其实现原理及技术,
能促进我们更好地理解计算机软硬件系统,提升我们的开发能力。
本文简要阐述了一个编译系统的实现过程,主要工作和成果包括:

(1)编译系统前端的设计与实现,能从文件中读取源代码生成使用结构体表示的抽象语法树;

(2)基于 Nanopass 思想,设计实现了编译系统后端,历经十余趟转换后生成 x86-64 汇编代码;

(3)将生成的汇编代码与 C 语言实现的运行时目标文件链接生成可执行文件,
运行时实现了垃圾回收等功能;

(4)最后针对本文实现的编译系统进行了详细的测试,保证了编译系统的可靠性。

实现的语言支持可变变量和整数运算,if 语句和 while 语句,头等函数和闭包,自动内存管理等。
虽然所有功能都得到了正确的实现,但仍然存在许多不足和需要改进的地方:

(1)增加语言的功能,例如结构体和模式匹配等,目前的语言非常简单,远远无法满足日常使用的需求;

(2)采用更先进的算法,加入更多优化,例如对闭包和垃圾回收算法的优化,
目前系统各个部分的算法采用的大多是比较简单的算法,且本文未与现有的编译系统进行对比,
没有对编译速度以及编译得到的程序大小与运行速度进行测试。
