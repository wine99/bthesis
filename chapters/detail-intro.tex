% !TeX root = ../main.tex

\chapter{目标平台介绍}

本章简要介绍 x86-64 汇编语言的相关知识。
我们首先通过一个简单的示例程序来讲解 x86-64 汇编语言的语法和过程调用栈,
然后介绍我们选择的调用约定和寄存器分配策略,
最后讲解如何在汇编层面为我们的函数式语言实现高效的尾调用。

\section{x86-64 程序示例}

本文使用 GNU 汇编器要求的 AT\&T x86-64 汇编语法。程序以 main 标签开始,后面跟着一系列指令。
globl 指令表明 main 过程是外部可见的,这是操作系统可以调用它的必要条件。
x86 程序存储在计算机的内存中。就本文的目的而言,计算机的内存是 64 位地址到 64 位值的映射。
计算机在 rip 寄存器中存储一个程序计数器 (PC) ,它指向下一条要执行的指令的地址。
对于大多数指令,程序计数器在指令执行后递增,因此它指向内存中的下一条指令。
大多数 x86 指令有两个操作数,每个操作数要么是一个整型常量 (称为立即数) ,
要么是一个寄存器 ,要么是一个内存位置。

寄存器是一种特殊的变量,我们能使用的是通用寄存器,共有16个,
每一个都有一个 64 位的值。
我们使用 \% 符号后跟寄存器名,比如 \code{\%rax},来指代某一个寄存器。

立即数使用语法 \code{\$n} 来指明,其中 n 是整数。
访问内存使用语法 \code{n(\%r)} 来指明,它的意思是拿到存储在寄存器 r 中的地址,然后给地址加 n 个字节。

我们来看一个简单但完整的 x86-64 程序示例。该程序计算式子\code{(+ 52 (- 10))}的值,
并将结果作为整个程序的返回值。

\begin{lstlisting}
start:
movq	$10, -8(%rbp)
negq	-8(%rbp)
movq	-8(%rbp), %rax
addq	$52, %rax
jmp conclusion

.globl main
main:
pushq	%rbp
movq	%rsp, %rbp
subq	$16, %rsp
jmp start
conclusion:
addq	$16, %rsp
popq	%rbp
retq
\end{lstlisting}

这个程序使用一个称为过程调用栈 (简称栈) 的内存区域。
堆栈由每个过程调用的独立帧组成。单个帧的内存布局如图\ref{fig:frame}所示。
寄存器 rsp 被称为堆栈指针 ,它指向堆栈顶部的项。
堆栈在内存中向下增长,通过减去堆栈指针来增加堆栈的大小。
在过程调用的上下文中,返回地址是调用方调用指令之后的指令。
函数调用指令 callq 在跳转到过程之前将返回地址压入堆栈。
寄存器 rbp 是 基指针 ,用于访问存储在当前过程调用帧中的变量。
调用者的基指针在返回地址之后被压入堆栈,然后基指针被设置为旧基指针的位置。
变量 1 存储在地址 −8(\%rbp) ,变量 2 存储在地址 −16(\%rbp) ,以此类推。

\begin{figure}[t]
	\centering
	\begin{tabular}{|r|l|} \hline
		位置 & 内容 \\ \hline
		8(\key{\%rbp}) & return address \\
		0(\key{\%rbp}) & old \key{rbp} \\
		-8(\key{\%rbp}) & variable $1$ \\
		-16(\key{\%rbp}) & variable $2$ \\
		\ldots  & \ldots \\
		0(\key{\%rsp}) & variable $n$\\ \hline
	\end{tabular}

	\caption{栈帧的内存布局}
	\label{fig:frame}
\end{figure}

回到这个程序,首先,操作系统发出一个 callq main 指令,
将其返回地址推送到堆栈上,然后跳转到 main 。
在 x86-64 中,堆栈指针 rsp 在执行任何 callq 指令之前必须整除 16 个字节。
因此当控制到达 main 时,rsp 偏离对齐 8 个字节 (因为 callq 压入返回地址)。

前三条指令是一个程序典型的准备工作。
指令 \code{pushq \%rbp} 将调用者的基指针保存到堆栈上,并从堆栈指针中减去 8 。
第二个指令 \code{movq \%rsp, \%rbp} 读取 rsp 中的内容,赋值给 rbp。
这条指令改变了基指针,使其指向旧基指针的位置。
指令 \code{subq \$16, \%rsp} 读取 rsp 中的内容,减去 16,再重新赋值给 rsp。
这条指令将堆栈指针向下移动,以便为存储变量留出足够的空间。

虽然这个程序只需要在栈上存储一个变量 (8 个字节) ,但是我们还是需要减去 16,
这样 rsp 是 16 字节对齐的,就可以调用其他函数。
准备工作的最后一条指令是 \code{jmp start},
它使程序跳转到由 Racket 表达式 \code{(+ 52 (- 10))} 生成的指令,也就是 start 标签下的指令。

start 标签下的第一条指令是 \code{movq \$10, -8(\%rbp)},
它将 10 存储在图\ref{fig:frame}中 variable 1 的位置。
指令 \code{negq -8(\%rbp)} 将值更改为 −10。下一条指令将 \%rax 中值更改为 -10。
最后,\code{addq \$52, \%rax} 将 52 加到 rax 上,将其值更新为 42。
我们总是约定将程序的返回值放在寄存器 rax 中。

conclusion 标签下的三个指令是一个程序典型的结束工作。
前两条指令将 rsp 和 rbp 寄存器恢复到过程开始时的状态。
指令 \code{addq \$16, \%rsp} 将栈指针移回指向旧的基指针。
然后 \code{popq \%rbp} 把旧基指针返回给 rbp,并在堆栈指针上加 8。
最后一条指令 retq 返回到调用它的过程,并在堆栈指针上加 8。

\section{x86 中的函数调用}
\label{sec:call-in-x86}

x86 提供标签,以便可以引用指令的位置,这是跳转指令所需要的。
标签也可以用来标记函数指令的开头。
我们还可以通过使用 leaq 指令和 PC 相对寻址来获得一个标签的地址。
例如,下面这条指令把 add 函数的地址放入 rbx 寄存器中:
\begin{lstlisting}
leaq add1(%rip), %rbx
\end{lstlisting}

指令指针寄存器 rip,也就是程序计数器始终指向要执行的下一条指令。
当与一个标签结合时,如\code{add1(\%rip)},
链接器计算 add1 标签的地址与 rip 此刻的位置之间的距离 d,
然后将 \code{add1(\%rip)} 更改为 \code{d(\%rip)},
通过这种方式在运行时计算出 add1 标签的地址。

由于支持一等函数,也就是将函数当作普通的变量,函数可能会被赋值给其他变量。
也就是说,某个变量中存放的可能是一个函数地址,调用这个变量就相当于调用函数,
亦即间接函数调用。对应的 x86 指令是接受一个寄存器名的 callq* 指令:
\begin{lstlisting}
callq *%rbx
\end{lstlisting}

callq 指令为实现函数提供了部分支持:它将返回地址压入堆栈,
然后跳转到目标函数。但是 callq 不处理:
\begin{enumerate}
\item 参数传递
\item 把帧压入过程调用栈以及取出或访问它们
\item 确定不同函数如何共享寄存器
\end{enumerate}

函数调用需要两段代码之间的协调,这两段代码可能由不同的程序员编写或由不同的编译器生成。
这里我们遵循 Linux 和 MacOS 上的 GNU C 编译器使用的
System V 调用约定\cite{Matz_Hubicka_Jaeger_Mitchell_2013}。
调用约定包括关于函数如何共享寄存器使用的规则。
特别是,调用方负责在函数调用之前释放一些寄存器,供被调用方使用。
这些被称为“调用者保存寄存器”,它们是
\begin{lstlisting}
rax rcx rdx rsi rdi r8 r9 r10 r11
\end{lstlisting}
另一方面,被调用方负责保留“被调用者保存寄存器”,即
\begin{lstlisting}
rsp rbp rbx r12 r13 r14 r15
\end{lstlisting}

我们可以从调用方视角和被调用方视角两个角度来考虑这个调用约定:
\begin{itemize}
\item 调用方应该假定所有调用者寄存器都可以被被调用方用任意值重写。
另一方面,调用者可以安全地假设所有调用者保存的寄存器在调用后包含与调用前相同的值。
\item 被调用方可以自由地使用调用者保存寄存器。
如果被调用方希望使用被调用者保存寄存器,
被调用方必须在返回到调用方之前将原始值放回寄存器中。
这可以通过在准备工作中将值保存到调用栈中,并在函数的结束工作中恢复来实现。
\end{itemize}

在 x86 中,寄存器还用于向函数传递参数和保存返回值。
函数的前六个参数按以下顺序,在以下六个寄存器中传递。

\begin{lstlisting}
rdi rsi rdx rcx r8 r9
\end{lstlisting}

多余的参数则保存在调用方的栈帧上。
寄存器 rax 则用于函数的返回值。

\section{高效的尾调用}
\label{sec:tail-call}

通常,程序所使用的栈空间的大小是由最长的嵌套函数调用链决定的。
也就是说,如果函数
$f_1$ 调用 $f_2$ , $f_2$ 调用 $f_3$, $\ldots$ , $f_{n-1}$ 调用$f_n$ ,
那么堆栈空间的大小就是 $O(n)$ 。
在递归或相互递归函数的情况下,深度$n$ 可以变得很大。
但是,如果函数调用表达式位于尾位置(尾位置的定义见\ref{sec:control}节),我们把它称为尾调用,它就不会导致调用栈的深度增加。
例如,在下面的求和函数中的递归调用就是尾调用。

\begin{lstlisting}
(define (sum [n : Integer] [result : Integer]) : Integer
  (if (eq? n 0)
      result
      (sum (- n 1) (+ result n))))
(sum 5 0)
\end{lstlisting}

尾调用时不再需要调用方的帧,因此可以在进行尾调用之前弹出调用方的帧。
通过这种方法,只进行尾调用的递归函数将只使用 $O(1)$ 的栈空间。
函数式语言严重依赖递归函数,所以要尾调用必须被优化。

弹出帧的指令也就是前面说的函数的结束工作。因此我们还需要在每个尾调用之前插入这些指令。
在将来尾调用完成时,它就不再需要也不能返回当前函数,而是直接返回到调用当前函数的那个函数。
因此尾调用不再使用callq指令,而是直接使用jmp指令进行跳转。
与间接函数调用类似,x86使用前面带一个星号的寄存器跳转指令来表示间接跳转:
\begin{lstlisting}
jmp *%rax
\end{lstlisting}
这里我们统一使用rax,因为跳转指令前面插入的那些结束工作指令会把一些数据恢复回到寄存器里,
这些数据将来还要被使用,不能覆盖掉它们。而rax此时总是空闲的。

另外还要注意的一点是,前文说过函数调用的前六个参数分别使用约定的六个寄存器,
多余的参数则使用过程调用栈来传递。
加入尾递归优化之后,用栈来传参就很困难了。因此在后面的编译过程中,
我们会把第六个及其后面的参数放入一个向量中,把这个向量作为第六个参数,以此来解决这个问题。
